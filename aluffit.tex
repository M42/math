% Created 2016-08-16 mar 23:57
\documentclass[11pt]{article}
\usepackage[utf8]{inputenc}
\usepackage[T1]{fontenc}
\usepackage{fixltx2e}
\usepackage{graphicx}
\usepackage{grffile}
\usepackage{longtable}
\usepackage{wrapfig}
\usepackage{rotating}
\usepackage[normalem]{ulem}
\usepackage{amsmath}
\usepackage{textcomp}
\usepackage{amssymb}
\usepackage{capt-of}
\usepackage[colorlinks=true]{hyperref}
\usepackage{amsthm}
\newtheorem{theorem}{Teorema}
\newtheorem{definition}{Definición}
\author{Mario Román}
\date{\today}
\title{Algebra: Chapter 0\\\medskip
\large Apuntes de teoría}
\hypersetup{
 pdfauthor={Mario Román},
 pdftitle={Algebra: Chapter 0},
 pdfkeywords={},
 pdfsubject={},
 pdfcreator={Emacs 25.1.50.2 (Org mode 8.3.4)}, 
 pdflang={Spanish}}
\begin{document}

\maketitle

\section*{IV. Álgebra lineal}
\label{sec:orgheadline4}
\subsection*{4. Presentaciones y resoluciones}
\label{sec:orgheadline3}
\subsubsection*{4.1. Torsión}
\label{sec:orgheadline1}
\begin{definition}
\textbf{Torsión}. Un elemento \(m \in M\) módulo de \(R\) es de \textbf{torsión} si \(\{m\}\) es linealmente
dependiente. Es decir,

\[ \exists r \in R,\ r \neq 0\ :\ rm = 0 \]

El conjunto de elementos de torsión se llama \(Tor(M)\). Un módulo es \textbf{libre de torsión}
si \(Tor(M) = 0\) y \textbf{de torsión} si \(Tor(M)=M\).
\end{definition}

Un anillo conmutativo es libre de torsión sobre sí mismo si y sólo si es dominio de
integridad. Cuando esto ocurre, \(Tor(M)\) es siempre submódulo de \(M\). Submódulos o
sumas de módulos libres de tensión serán libres de torsión, y por todo esto, los módulos
libres sobre dominios de integridad serán libres de torsión.

\begin{definition}
\textbf{Cíclico}. Un módulo es \textbf{cíclico} cuando es generado por un elemento. Es decir,
cuando \(M \cong R/I\) para algún ideal.
\end{definition}

La equivalencia se ve en este \href{aluffi.org}{ejercicio}. Cuando en un dominio de integridad todos sus
módulos cíclicos son libres de torsión, es un cuerpo. Otra forma de pensar sobre un módulo
cíclico es como aquel que admite un epimorfismo:

\[ R \rightarrow M \rightarrow 0 \]

\subsubsection*{4.2.}
\label{sec:orgheadline2}
\end{document}
